
\usepackage[margin=2.5cm]{geometry}

\usepackage{setspace}

%\geometry{
%	paper=a4paper, % Change to letterpaper for US letter
%	inner=2.5cm, % Inner margin
%	outer=3.8cm, % Outer margin
%	bindingoffset=.5cm, % Binding offset
%	top=1.5cm, % Top margin
%	bottom=1.5cm, % Bottom margin
%	%showframe, % Uncomment to show how the type block is set on the page
%}


\usepackage{natbib}
%\usepackage[utf8]{inputenc} % allow utf-8 input
%\usepackage[T1]{fontenc}    % use 8-bit T1 fonts
\usepackage{hyperref}       % hyperlinks
\usepackage{url}            % simple URL typesetting
\usepackage{booktabs}       % professional-quality tables
\usepackage{amsfonts}       % blackboard math symbols
\usepackage{nicefrac}       % compact symbols for 1/2, etc.
\usepackage{microtype}      % microtypography
\usepackage{fancyhdr}
\pagestyle{fancy}

%\usepackage{algorithm}
\usepackage{algorithmic}
\usepackage{algorithm2e}


\usepackage{amsfonts}
\usepackage{amsmath}
\usepackage{amsthm}
\usepackage{amssymb}
\usepackage{dsfont}
\usepackage{enumitem}
\usepackage{filecontents}
\usepackage{graphicx}
\usepackage{subcaption}
\usepackage{wrapfig}
\usepackage{bm}
\usepackage{color}


\newtheorem{assumption}{Assumption}
\newtheorem{definition}{Definition}
\newtheorem{theorem}{Theorem}
\newtheorem{lemma}{Lemma}
\newtheorem{proposition}{Proposition}
\newtheorem{corollary}{Corollary}
\newtheorem{remark}{Remark}
%\newcommand*{\qed}{\hfill\ensuremath{\square}}%

% Standard math operations and sets
\newcommand{\diff}{\mathop{}\!\mathrm{d}}
\DeclareMathOperator*{\argmin}{argmin}
\DeclareMathOperator*{\argmax}{argmax}
\newcommand{\PrNone}{\mathbb{P}}
\renewcommand{\Pr}[1]{\PrNone\left( #1 \right)}
\newcommand{\ENone}{\mathds{E}}
\newcommand{\EIN}[1]{\ENone\left[ #1 \right]}
\newcommand{\KL}[2]{\operatorname{KL}\left( #1 \| #2 \right)}
\newcommand{\R}{\mathbb{R}}
\newcommand{\N}{\mathbb{N}}
\newcommand{\1}[1]{\mathds{1}_{\left( #1 \right)}}
\newcommand{\Set}[1]{\mathchoice%
{\left\{ #1 \right\}}{\{ #1 \}}{\{ #1 \}}{\{ #1 \}}}

% Symbols used in the paper
\newcommand{\BernKL}[2]{\operatorname{kl}\left( #1, #2 \right)} % Bernoulli KL divergence

% Problem \ Lower bound symbols
\newcommand{\historyRV}{H}
\newcommand{\armRV}{W}
\newcommand{\rewardRV}{Z}

\newcommand{\Arms}{\mathcal{W}}
\newcommand{\arm}{w}
\newcommand{\ArmEvents}{\mathcal{F}_\Arms}
\newcommand{\ArmMeasure}{M}
\newcommand{\ArmMean}{\mu}
\newcommand{\ArmMeanAlt}{\lambda}
\newcommand{\ArmGrp}[1]{\mathcal{S}^{#1}}
\newcommand{\ArmGrpEmp}[1]{\mathcal{S}_{#1}}
\newcommand{\ArmKLNone}{\operatorname{d}}
\newcommand{\ArmKL}[2]{\ArmKLNone\hspace{-.1em}\left( #1, #2 \right)}
\newcommand{\ArmCI}[2]{\operatorname{d^*}\hspace{-.2em}\left( #1, #2 \right)}

\newcommand{\ExpRewards}{\Theta}
\newcommand{\expReward}{\theta}
\newcommand{\ExpRewardEvents}{\mathcal{F}_\ExpRewards}
\newcommand{\ExpRewardMeasure}{M_\ExpRewards}
\newcommand{\ExpRewardDensity}{g}
\newcommand{\ExpRewardCDF}{G}
\newcommand{\ExpRewardInvCDF}{G^{-1}}

\newcommand{\GTargetEps}{\mathcal{G}_{\ArmMeasure,\ArmMean}^{\alpha,\epsilon}}
\newcommand{\GTargetZeroEps}{\mathcal{G}_{\ArmMeasure,\ArmMean}^{\alpha,0}}
\newcommand{\GTargetNoEps}{\mathcal{G}_{\ArmMeasure,\ArmMean}^{\alpha}}
\newcommand{\GTargetNoEpsAlt}{\mathcal{G}_{\ArmMeasure,\ArmMeanAlt^i}^{\alpha}}
\newcommand{\GTargetNoEpsAltOne}{\mathcal{G}_{\ArmMeasure,\ArmMeanAlt^1}^{\alpha}}

\newcommand{\PrTrueModelNone}{\PrNone_{\ArmMean}^\ArmMeasure}
\newcommand{\PrTrueModel}[1]{\PrNone_{\ArmMean}^\ArmMeasure\left( #1 \right)}
\newcommand{\PrAltModel}[1]{\PrNone_{\ArmMeanAlt}^\ArmMeasure\left( #1 \right)}
\newcommand{\PrAltModelI}[1]{\PrNone_{\ArmMeanAlt^i}^\ArmMeasure\left( #1 \right)}
\newcommand{\PrAltModelOne}[1]{\PrNone_{\ArmMeanAlt^1}^\ArmMeasure\left( #1 \right)}

\newcommand{\ETrueModelNone}{\ENone_{\ArmMean}^\ArmMeasure}
\newcommand{\ETrueModel}[1]{\ENone_{\ArmMean}^\ArmMeasure\left[ #1 \right]}
\newcommand{\EAltModel}[1]{\ENone_{\ArmMeanAlt}^\ArmMeasure\left[ #1 \right]}

% Algorithm \ Upper bound symbols
\newcommand{\ArmInterval}{\mathcal{I}_{a}(t)}

% Upper bound proof symbols
\newcommand{\EpsCheapArms}{\mathcal{A}_{\epsilon}}
%\newcommand{\cE}{\mathcal{E}}
%\newcommand{\cF}{\mathcal{F}}
%\newcommand{\cR}{\mathcal{R}}



\renewcommand{\algorithmicrequire}{\textbf{Input:}}
\renewcommand{\algorithmicensure}{\textbf{Output:}}
\renewcommand{\algorithmiccomment}[1]{// #1}

\usepackage{bm}

%%
%% This is Jay's ``implication array'' format.
% %%

\catcode`@=11

%%%  Implication arrays
%
% Usage:
%
% \begin{imparray}
% (implication) & (left side) & (comparator) & (right side) \\
% ...
% \end{imparray}

\def\imparray{\stepcounter{equation}\let\@currentlabel=\theequation
\global\@eqnswtrue
\global\@eqcnt\z@\tabskip\@centering\let\\=\@eqncr
$$\halign to \displaywidth\bgroup\llap{${##}$\hskip 4\arraycolsep}\tabskip\z@&
  \@eqnsel\hskip\@centering
  $\displaystyle\tabskip\z@{##}$&\global\@eqcnt\@ne 
  \hskip 2\arraycolsep \hfil${##}$\hfil
  &\global\@eqcnt\tw@ \hskip 2\arraycolsep $\displaystyle\tabskip\z@{##}$\hfil 
   \tabskip\@centering&\llap{##}\tabskip\z@\cr}

\def\endimparray{\@@eqncr\egroup
      \global\advance\c@equation\m@ne$$\global\@ignoretrue}

\@namedef{imparray*}{\def\@eqncr{\nonumber\@seqncr}\imparray}
\@namedef{endimparray*}{\nonumber\endimparray}

\catcode`@=12


% from fixed buddget paper

%\usepackage[margin=1.5in]{geometry}

% \usepackage{algorithm2e}
% \usepackage{enumitem}
% \usepackage{filecontents}
% \usepackage{todonotes}
% \usepackage{graphicx}
%\usepackage{subcaption}
% \usepackage{longtable}
% \usepackage{booktabs}
%\usepackage{wrapfig}
% \usepackage[numbers]{natbib}
%\usepackage{algorithm}
%\usepackage{algorithmic}

%\usepackage[margin=1in]{geometry}
\newcommand{\kevin}[1]{{\textcolor{red}{#1}}}
\newcommand\tab[1][1cm]{\hspace*{#1}}
\newcommand{\polylog}{{\rm polylog}}

\def\calD{\mathcal{D}}
\def\calE{\mathcal{E}}
\def\H{\mathcal{H}}
\def\calS{\mathcal{S}}
\def\calI{\mathcal{I}}
\def\S{\mathcal{S}}
\def\A{\mathcal{A}}
\def\E{\mathbb{E}}
\def\1{\mathbf{1}}
\def\P{\mathbb{P}}
\def\R{\mathbb{R}}
\def\mualpha{\mu^{(\alpha)}}
\newcommand{\mc}[1]{\mathcal{#1}}
\newcommand{\mbb}[1]{\mathbb{#1}}
\newcommand{\mbf}[1]{\mathbf{#1}}

%\newtheorem{theorem}{Theorem}
%\newtheorem{lemma}{Lemma}
%\newtheorem{remark}{Remark}
%\newtheorem{assumption}{Assumption}
%\newtheorem{definition}{Definition}


%from dosefinding paper

\def\bSig\mathbf{\Sigma}

\usepackage{rotating}
%\usepackage{dosefinding/macrosArticle}
%\usepackage{ulem}
\usepackage[normalem]{ulem}
\usepackage{xcolor}
\usepackage{booktabs}

\newcommand{\MTD}{\mathrm{MTD}}
\newcommand{\Opt}{\mathrm{Opt}}

%\usepackage[figuresright]{rotating}
\newcommand{\limInf}{\underline{\lim}}
\newcommand{\eff}{\text{eff}}
\newcommand{\tox}{\text{tox}}
\newcommand{\acrm}{\hat{a}_{\mathrm{CRM}}(t)}
\newcommand{\aTS}{\tilde{a}_{\mathrm{TS}}(t)}
\newcommand{\TSOne}{\mathrm{TS}\_\mathrm{V1}}
\newcommand{\TSTwo}{\mathrm{TS}\_\mathrm{V2}}
%\newcommand{\Set}[1]{\mathchoice%
%{\left\{ #1 \right\}}{\{ #1 \}}{\{ #1 \}}{\{ #1 \}}}
%\newcommand{\E}{\mathds{E}}
\renewcommand{\P}{\mathbb{P}}
\newcommand{\kl}{\mathrm{kl}}
\newcommand{\ymid}{y_{\mathrm{mid}}}

\newcommand{\tblopt}[1]{\underline{#1}} % Mark optimal dose in table
\newcommand{\tblwinrec}[1]{\textbf{#1}} % Mark when opt dose recommended more than baseline

\usepackage{tikz}

\newcommand{\dash}[1]{%
    \tikz[baseline=(todotted.base)]{
        \node[inner sep=1pt,outer sep=0pt] (todotted) {#1};
        \draw[dashed] ([yshift=-2pt]todotted.south west) -- ([yshift=-2pt]todotted.south east);
    }%
}%

% Define numbers used for Web Appendices and Web Tables
%\def \refWAProofSH {Web Appendix A}
%\def \refWAProofTS {Web Appendix B}
%\def \refWAResults {Web Appendix C}
%\def \refWTTox {Web Table 1}
%\def \refWTEffa {Web Table 2}
%\def \refWTEffb {Web Table 3}
%\def \refWTEffs {Web Tables 2 and 3}



%% Language
%\usepackage[utf8]{inputenc}
%\usepackage[english]{babel}
%\usepackage{dsfont}

%% Other useful packages
%\usepackage{algorithm}
%\usepackage{algorithmicx}
%\usepackage[noend]{algpseudocode}

%%\usepackage{algorithm2e}
%\usepackage{xcolor}
%%\usepackage{subfigure}
%\usepackage{verbatim}
%\usepackage{times}
%\usepackage{graphicx}
%\DeclareGraphicsExtensions{.jpg,.pdf,.mps,.eps,.png}
%%\usepackage{amsthm,amscd}
%\usepackage{amsmath,amssymb}
%\usepackage{pdfpages}
%\usepackage{bm}%bold maths symbols
%\usepackage{amsfonts}
%%\usepackage{MnSymbol}
%\usepackage{todonotes}

%% Margins
%\oddsidemargin=0pt
%\textwidth=455pt
%\textheight=620pt
%\voffset=-20pt
%\marginparwidth=0pt
%\marginparpush=0pt
%\marginparsep=0pt
%\evensidemargin=0pt


%% Color definition
%\usepackage{xcolor}
%\definecolor{Bleu}{RGB}{0,0,204}
%\definecolor{Violet}{RGB}{102,0,204}
%\definecolor{Rouge}{RGB}{204,0,0}
%\definecolor{Highlight}{RGB}{251,0,0}


%% Bibliography and references
%\bibliographystyle{apalike}

%\usepackage{hyperref}
%\hypersetup{
%colorlinks,
%  citecolor=Bleu,
%  linkcolor=Rouge,
%  urlcolor=Violet} 

\usepackage{breakcites} % repare pb de mise à la ligne des references

  
 
  
  
%% Theorem English
%\newtheorem{theorem}{Theorem}%[section]
%\newtheorem{assumption}[theorem]{Assumption}
%\newtheorem{claim}[theorem]{Claim}
%\newtheorem{corollary}[theorem]{Corollary}
%\newtheorem{definition}[theorem]{Definition}
%\newtheorem{example}[theorem]{Example}
%\newtheorem{lemma}[theorem]{Lemma}
%\newtheorem{notation}[theorem]{Notation}
%\newtheorem{proposition}[theorem]{Proposition}
%\newtheorem{remark}[theorem]{Remark}


%Calligraphic Shorthands
\newcommand{\cA}{\mathcal{A}}
\newcommand{\cB}{\mathcal{B}}
\newcommand{\cC}{\mathcal{C}}
\newcommand{\cD}{\mathcal{D}}
\newcommand{\cE}{\mathcal{E}}
\newcommand{\cF}{\mathcal{F}}
\newcommand{\cG}{\mathcal{G}}
\newcommand{\cH}{\mathcal{H}}
\newcommand{\cI}{\mathcal{I}}
\newcommand{\cJ}{\mathcal{J}}
\newcommand{\cK}{\mathcal{K}}
\newcommand{\cL}{\mathcal{L}}
\newcommand{\cM}{\mathcal{M}}
\newcommand{\cN}{\mathcal{N}}
\newcommand{\cO}{\mathcal{O}}
\newcommand{\cP}{\mathcal{P}}
\newcommand{\cQ}{\mathcal{Q}}
\newcommand{\cR}{\mathcal{R}}
\newcommand{\cS}{\mathcal{S}}
\newcommand{\cT}{\mathcal{T}}
\newcommand{\cU}{\mathcal{U}}
\newcommand{\cV}{\mathcal{V}}
\newcommand{\cW}{\mathcal{W}}
\newcommand{\cX}{\mathcal{X}}
\newcommand{\cY}{\mathcal{Y}}
\newcommand{\cZ}{\mathcal{Z}}

%Blackboard Bold Shorthands
\newcommand{\bE}{\mathbb{E}}
\newcommand{\bP}{\mathbb{P}}
%\newcommand{\N}{\mathbb{N}}
%\newcommand{\R}{\mathbb{R}}
%\newcommand{\C}{\mathbb{C}}
%\newcommand{\Z}{\mathbb{Z}}

%Probability Shorthands
\def \ind{\mathds{1}}
\newcommand{\norm}[2]{\mathcal{N}\left(#1,#2\right)}
\newcommand{\BetaD}[2]{\text{Beta}\left(#1,#2 \right)}
\newcommand{\muhat}{\hat{\mu}}


% Bandit specific shorthands
\newcommand{\reg}{\text{R}}
\newcommand{\regfull}{\reg_{\bm{\theta}}(T,\cA)}
\newcommand{\br}{\text{BR}}
\newcommand{\Kinf}{\cK_{\text{inf}}\,}
\newcommand{\muA}{\hat{\mu}_a(t)}%moyenne empirique dans l'algorithme 
\newcommand{\muS}{\hat{\mu}_{a,s}}
\newcommand{\thetaU}{\underline{\theta}}

% Statistical Shorthands
\newcommand{\K}{\mathrm{K}}
\newcommand{\Kb}{\mathrm{K}}


% Misc Math Operations Shorthands
\newcommand{\argminin}[1]{\underset{#1}{\text{argmin }}}
\newcommand{\argmaxin}[1]{\underset{#1}{\text{argmax }}}
\newcommand{\ngoesto}{\underset{n\rightarrow \infty}{\rightarrow}}
\newcommand{\Tgoesto}{\underset{T\rightarrow \infty}{\longrightarrow}}
\newcommand{\isequT}{\underset{T\rightarrow \infty}{\sim}}

%\def \qed {\begin{flushright} $\qquad \Box $
%  \end{flushright}}



% Coloring and working together
\def \red{\color{red}}
\def \black{\color{black}}

% from DT paper

%\usepackage[utf8]{inputenc} % allow utf-8 input
%\usepackage[T1]{fontenc}    % use 8-bit T1 fonts
%\usepackage{hyperref}       % hyperlinks
%\usepackage{url}            % simple URL typesetting
%\usepackage{booktabs}       % professional-quality tables
%\usepackage{amsfonts}       % blackboard math symbols
%\usepackage{nicefrac}       % compact symbols for 1/2, etc.
%\usepackage{microtype}      % microtypography
%
%\usepackage{graphicx}
%\usepackage{algorithm}
%\usepackage{algorithmic}
%
%% Attempt to make hyperref and algorithmic work together better:
%%\newcommand{\theHalgorithm}{\arabic{algorithm}}
%
%\usepackage[title]{appendix}
%\usepackage{afterpage}
%\usepackage{amsfonts}
%\usepackage{amsmath}
%\usepackage{amsthm}
%\usepackage{amssymb}
%\usepackage{dsfont}
%\usepackage{enumitem}
%\usepackage{filecontents}
%\usepackage{todonotes}
%\usepackage{graphicx}
%\usepackage{subcaption}
%\usepackage{wrapfig}
\usepackage{bbm}
%
%\newtheorem{assumption}{Assumption}
%\newtheorem{definition}{Definition}
%\newtheorem{theorem}{Theorem}
%\newtheorem{lemma}{Lemma}
%\newtheorem{proposition}{Proposition}
%\newtheorem{corollary}{Corollary}
%\newtheorem{remark}{Remark}
%
\newcommand{\bP}{\mathbb{P}}
%\newcommand{\1}[1]{\mathds{1}{\left\{ #1 \right\}}}
%
%\DeclareMathOperator*{\argmin}{argmin}
%
%\newcommand{\Set}[1]{\mathchoice{\left\{ #1 \right\}}{\{ #1 \}}{\{ #1 \}}{\{ #1 \}}}




