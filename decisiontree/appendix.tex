\subsection{Additional Results}
\subsubsection{Train and Test Error for AdaBoost}
Table~\ref{tbl:AdaBoostTestTrainErrors} reports test and train errors at various Boosting rounds. Our algorithm achieves the test and train error in fewer total number of example assessments, compared to \texttt{Quick Boost}. Note that both algorithms, except in the case of RCV1, have the same test and train error at a given round, as they should because both train identical decision trees. The case of RCV1 is due to the algorithms picking a weak learner arbitrarily in case of ties, without changing the overall results significantly.
\begin{table*}[ht]
\caption{AdaBoost results,
reported at rounds 100, 300 and 500 (400 for RCV1).}
\label{tbl:AdaBoostTestTrainErrors}
\vskip 0.15in
\begin{center}
\begin{small}
\begin{sc}
\begin{tabular}{lccccccccc}
\toprule
	& \multicolumn{3}{c}{100}
	& \multicolumn{3}{c}{300}
	& \multicolumn{3}{c}{400/500}
	\\
Alg: Data & \# Assess. & Train & Test & \# Assess. & Train & Test & \# Assess. & Train & Test \\
\midrule
AP-B: a6a & 3.35e+08 & 0.142 & 0.155 & 1.02e+09 & 0.131 & 0.157 & 1.69e+09 & 0.128 & 0.160 \\
QB: a6a & 3.57e+08 & 0.142 & 0.155 & 1.09e+09 & 0.131 & 0.157 & 1.83e+09 & 0.128 & 0.160 \\
AP-B: mnist & 1.26e+11 & 0.106 & 0.111 & 3.52e+11 & 0.057 & 0.064 & --- & --- & --- \\
QB: mnist & 1.36e+11 & 0.106 & 0.111 & 3.96e+11 & 0.057 & 0.064 & --- & --- & --- \\
AP-B: rcv1 & 1.73e+11 & 0.027 & 0.059 & 4.83e+11 & 0.005 & 0.047 & 6.15e+11 & 0.001 & 0.044 \\
QB: rcv1 & 1.85e+11 & 0.029 & 0.061 & 5.13e+11 & 0.004 & 0.047 & 6.58e+11 & 0.001 & 0.046 \\
AP-B: satimage & 1.98e+08 & 0.113 & 0.150 & 5.46e+08 & 0.070 & 0.121 & 8.64e+08 & 0.049 & 0.109 \\
QB: satimage & 2.20e+08 & 0.113 & 0.150 & 6.61e+08 & 0.070 & 0.121 & 1.11e+09 & 0.049 & 0.109 \\
AP-B: w4a & 3.92e+08 & 0.011 & 0.019 & 1.07e+09 & 0.006 & 0.018 & 1.69e+09 & 0.006 & 0.018 \\
QB: w4a & 4.64e+08 & 0.011 & 0.020 & 1.45e+09 & 0.006 & 0.018 & 2.41e+09 & 0.006 & 0.018 \\
\bottomrule
\end{tabular}
\end{sc}
\end{small}
\end{center}
\vskip -0.1in
\end{table*}


\subsection{Train and Test Error for LazyBoost and Weight Trimming}
\begin{table*}[ht]
\caption{Performance for A6A}
\label{tbl:perf-a6a}
\vskip 0.15in
\begin{center}
\begin{small}
\begin{sc}
\begin{tabular}{lccccccccc}
\toprule
	& \multicolumn{3}{c}{100}
	& \multicolumn{3}{c}{300}
	& \multicolumn{3}{c}{500}
	\\
 & \# Assess. & Train & Test & \# Assess. & Train & Test & \# Assess. & Train & Test \\
\midrule
AP LazyBoost (0.5) & 1.69e+08 & 0.145 & 0.156 & 5.11e+08 & 0.134 & 0.159 & 8.48e+08 & 0.129 & 0.160 \\
QB LazyBoost (0.5) & 1.80e+08 & 0.145 & 0.157 & 5.50e+08 & 0.137 & 0.158 & 9.22e+08 & 0.132 & 0.160 \\
AP LazyBoost (0.9) & 2.99e+08 & 0.141 & 0.156 & 9.07e+08 & 0.133 & 0.157 & 1.51e+09 & 0.130 & 0.159 \\
QB LazyBoost (0.9) & 3.18e+08 & 0.141 & 0.156 & 9.75e+08 & 0.133 & 0.157 & 1.64e+09 & 0.130 & 0.159 \\
AP Wt. Trim (0.9) & 2.45e+08 & 0.151 & 0.157 & 7.35e+08 & 0.151 & 0.157 & 1.23e+09 & 0.151 & 0.157 \\
QB Wt. Trim (0.9) & 2.49e+08 & 0.151 & 0.157 & 7.46e+08 & 0.151 & 0.157 & 1.24e+09 & 0.151 & 0.157 \\
AP Wt. Trim (0.99) & 3.16e+08 & 0.141 & 0.156 & 9.34e+08 & 0.132 & 0.157 & 1.54e+09 & 0.126 & 0.158 \\
QB Wt. Trim (0.99) & 3.28e+08 & 0.141 & 0.156 & 9.62e+08 & 0.132 & 0.157 & 1.58e+09 & 0.126 & 0.160 \\
\bottomrule
\end{tabular}
\end{sc}
\end{small}
\end{center}
\vskip -0.1in
\end{table*}

\begin{table*}[ht]
\caption{Performance for MNIST Digits}
\label{tbl:perf-mnist}
\vskip 0.15in
\begin{center}
\begin{small}
\begin{sc}
\begin{tabular}{lccccccccc}
\toprule
	& \multicolumn{3}{c}{100}
	& \multicolumn{3}{c}{200}
	& \multicolumn{3}{c}{300}
	\\
 & \# Assess. & Train & Test & \# Assess. & Train & Test & \# Assess. & Train & Test \\
\midrule
AP LazyBoost (0.5) & 6.65e+10 & 0.150 & 0.145 & 1.28e+11 & 0.098 & 0.098 & 1.87e+11 & 0.076 & 0.079 \\
QB LazyBoost (0.5) & 7.07e+10 & 0.150 & 0.145 & 1.39e+11 & 0.098 & 0.098 & 2.07e+11 & 0.076 & 0.079 \\
AP LazyBoost (0.9) & 1.17e+11 & 0.117 & 0.118 & 2.22e+11 & 0.079 & 0.085 & 3.20e+11 & 0.061 & 0.069 \\
QB LazyBoost (0.9) & 1.25e+11 & 0.117 & 0.118 & 2.43e+11 & 0.079 & 0.085 & 3.59e+11 & 0.061 & 0.069 \\
AP Wt. Trim (0.9) & 1.53e+11 & 0.901 & 0.901 & 3.07e+11 & 0.901 & 0.901 & 4.61e+11 & 0.901 & 0.901 \\
QB Wt. Trim (0.9) & 1.53e+11 & 0.900 & 0.901 & 3.07e+11 & 0.900 & 0.901 & 4.61e+11 & 0.900 & 0.901 \\
AP Wt. Trim (0.99) & 1.19e+11 & 0.117 & 0.124 & 2.21e+11 & 0.076 & 0.080 & 3.16e+11 & 0.062 & 0.068 \\
QB Wt. Trim (0.99) & 1.29e+11 & 0.115 & 0.117 & 2.37e+11 & 0.074 & 0.078 & 3.37e+11 & 0.056 & 0.061 \\
\bottomrule
\end{tabular}
\end{sc}
\end{small}
\end{center}
\vskip -0.1in
\end{table*}


\begin{table*}[ht]
\caption{Performance for RCV1}
\label{tbl:perf-rcv1}
\vskip 0.15in
\begin{center}
\begin{small}
\begin{sc}
\begin{tabular}{lccccccccc}
\toprule
	& \multicolumn{3}{c}{100}
	& \multicolumn{3}{c}{300}
	& \multicolumn{3}{c}{400}
	\\
 & \# Assess. & Train & Test & \# Assess. & Train & Test & \# Assess. & Train & Test \\
\midrule
AP LazyBoost (0.5) & 8.93e+10 & 0.029 & 0.061 & 2.48e+11 & 0.006 & 0.047 & 3.18e+11 & 0.002 & 0.046 \\
QB LazyBoost (0.5) & 9.06e+10 & 0.028 & 0.060 & 2.55e+11 & 0.005 & 0.048 & 3.29e+11 & 0.002 & 0.046 \\
AP LazyBoost (0.9) & 1.59e+11 & 0.027 & 0.058 & 4.35e+11 & 0.005 & 0.047 & 5.60e+11 & 0.002 & 0.045 \\
QB LazyBoost (0.9) & 1.64e+11 & 0.027 & 0.058 & 4.62e+11 & 0.004 & 0.047 & 5.93e+11 & 0.001 & 0.045 \\
AP Wt. Trim (0.9) & 1.19e+11 & 0.022 & 0.059 & 2.92e+11 & 0.003 & 0.047 & 3.65e+11 & 0.001 & 0.046 \\
QB Wt. Trim (0.9) & 1.22e+11 & 0.025 & 0.058 & 3.03e+11 & 0.003 & 0.047 & 3.79e+11 & 0.001 & 0.046 \\
AP Wt. Trim (0.99) & 1.62e+11 & 0.027 & 0.059 & 4.40e+11 & 0.004 & 0.047 & 5.61e+11 & 0.001 & 0.045 \\
QB Wt Trim (0.99) & 1.70e+11 & 0.027 & 0.059 & 4.60e+11 & 0.004 & 0.048 & 5.86e+11 & 0.001 & 0.046 \\
\bottomrule
\end{tabular}
\end{sc}
\end{small}
\end{center}
\vskip -0.1in
\end{table*}


\begin{table*}[ht]
\caption{Performance for SATIMAGE}
\label{tbl:perf-satimage}
\vskip 0.15in
\begin{center}
\begin{small}
\begin{sc}
\begin{tabular}{lccccccccc}
\toprule
	& \multicolumn{3}{c}{100}
	& \multicolumn{3}{c}{300}
	& \multicolumn{3}{c}{500}
	\\
 & \# Assess. & Train & Test & \# Assess. & Train & Test & \# Assess. & Train & Test \\
\midrule
AP LazyBoost (0.5) & 1.11e+08 & 0.133 & 0.152 & 3.22e+08 & 0.094 & 0.123 & 5.17e+08 & 0.073 & 0.115 \\
QB LazyBoost (0.5) & 1.23e+08 & 0.130 & 0.150 & 3.68e+08 & 0.090 & 0.129 & 6.11e+08 & 0.067 & 0.113 \\
AP LazyBoost (0.9) & 1.88e+08 & 0.114 & 0.128 & 5.13e+08 & 0.071 & 0.119 & 8.05e+08 & 0.050 & 0.110 \\
QB LazyBoost (0.9) & 2.06e+08 & 0.114 & 0.128 & 6.07e+08 & 0.071 & 0.119 & 1.01e+09 & 0.050 & 0.110 \\
AP Wt. Trim (0.9) & 2.51e+08 & 0.756 & 0.766 & 7.56e+08 & 0.756 & 0.766 & 1.26e+09 & 0.756 & 0.766 \\
QB Wt. Trim (0.9) & 2.51e+08 & 0.755 & 0.765 & 7.57e+08 & 0.755 & 0.765 & 1.26e+09 & 0.755 & 0.765 \\
AP Wt. Trim (0.99) & 1.80e+08 & 0.109 & 0.141 & 4.66e+08 & 0.066 & 0.121 & 7.01e+08 & 0.045 & 0.113 \\
QB Wt. Trim (0.99) & 1.89e+08 & 0.109 & 0.141 & 4.91e+08 & 0.066 & 0.121 & 7.39e+08 & 0.045 & 0.113 \\
\bottomrule
\end{tabular}
\end{sc}
\end{small}
\end{center}
\vskip -0.1in
\end{table*}

\begin{table*}[ht]
\caption{Performance for W4A}
\label{tbl:perf-w4a}
\vskip 0.15in
\begin{center}
\begin{small}
\begin{sc}
\begin{tabular}{lccccccccc}
\toprule
	& \multicolumn{3}{c}{100}
	& \multicolumn{3}{c}{300}
	& \multicolumn{3}{c}{500}
	\\
 & \# Assess. & Train & Test & \# Assess. & Train & Test & \# Assess. & Train & Test \\
\midrule
AP LazyBoost (0.5) & 2.00e+08 & 0.012 & 0.019 & 5.46e+08 & 0.008 & 0.018 & 8.61e+08 & 0.006 & 0.018 \\
QB LazyBoost (0.5) & 2.35e+08 & 0.012 & 0.019 & 7.35e+08 & 0.008 & 0.018 & 1.22e+09 & 0.006 & 0.018 \\
AP LazyBoost (0.9) & 3.48e+08 & 0.012 & 0.020 & 9.66e+08 & 0.007 & 0.018 & 1.52e+09 & 0.006 & 0.018 \\
QB LazyBoost (0.9) & 4.27e+08 & 0.012 & 0.020 & 1.32e+09 & 0.007 & 0.018 & 2.19e+09 & 0.006 & 0.018 \\
AP Wt. Trim (0.9) & 2.87e+08 & 0.016 & 0.021 & 8.41e+08 & 0.016 & 0.021 & 1.40e+09 & 0.016 & 0.021 \\
QB Wt. Trim (0.9) & 2.97e+08 & 0.016 & 0.021 & 8.63e+08 & 0.016 & 0.021 & 1.43e+09 & 0.016 & 0.021 \\
AP Wt. Trim (0.99) & 3.63e+08 & 0.012 & 0.020 & 9.44e+08 & 0.007 & 0.017 & 1.44e+09 & 0.006 & 0.018 \\
QB Wt. Trim (0.99) & 3.96e+08 & 0.012 & 0.020 & 1.01e+09 & 0.007 & 0.018 & 1.52e+09 & 0.006 & 0.018 \\
\bottomrule
\end{tabular}
\end{sc}
\end{small}
\end{center}
\vskip -0.1in
\end{table*}


\clearpage
\subsubsection {Different Tree Depths}
\begin{table*}[ht]
\caption{Different Tree Depths: Number of Assessments after 500 rounds}
\label{DiffDepths}
\vskip 0.15in
\begin{center}
\begin{small}
\begin{sc}
\begin{tabular}{llccccc}
\toprule
 &  & 1 & 2 & 3 & 4 & 5\\
\midrule
a6a & AP Boost & 6.40e+08 & 1.23e+09 & 1.69e+09 & 2.08e+09 & 2.44e+09 \\
a6a & Quick Boost & 6.66e+08 & 1.29e+09 & 1.83e+09 & 2.34e+09 & 2.89e+09 \\
\midrule
w4a & AP Boost & 8.71e+08 & 1.38e+09 & 1.69e+09 & 1.90e+09 & 2.12e+09 \\
w4a & Quick Boost & 9.10e+08 & 1.72e+09 & 2.41e+09 & 3.07e+09 & 3.60e+09 \\
\bottomrule
\end{tabular}
\end{sc}
\end{small}
\end{center}
\vskip -0.1in
\end{table*}

We also experimented with different tree depths, and found that \texttt{Adaptive-Pruning Boost} shows more dramatic gains in terms of total number of assessments when it uses deeper trees as weak learners.
We believe this is because of accumulated gains for training more nodes
in each tree. 
We have included an example of this in Table~\ref{DiffDepths}, where for two datasets (W4A, and A6A) we show experiments at depth 1 through 5. We report the total number of assessments used by \texttt{AdaBoost} (exact greedy-optimal decision trees) after 500 rounds.

\clearpage

%\input{appendix_infogain.tex}
%\input{appendix_infogainTight.tex}
%\input{appendix_gini}
