\subsection{Setup and Notation}
We adopt the setup, 
description and notation of~\citet{icml2013_appel13} for ease of
comparison.  

\paragraph {A Generic Boosting Algorithm.}
Boosting algorithms train a linear combination of classifiers
$\mathcal{H}_T(x)=\sum^T_t {\alpha_t h_t(x)}$
such that an error function $\mathcal{E}$ is minimized by optimizing scalar
$\alpha_t$ and the weak learner $h_t(x)$ at round $t$.
Examples $x_i$ misclassified by $h_t(x)$ are assigned ``heavy'' weights $w_i$
so that the algorithm focuses on these heavy weight examples when training weak
learner $h_{t+1}(x)$ in round $t+1$.
Decision trees, defined formally below, are often used as weak learners.

\paragraph {Decision Tree.}  A binary decision tree $h_{\textit{Tree}}(x)$ is a
tree-based classifier where every non-leaf node is a decision stump
$h(x)$.  A decision stump can be viewed as a tuple $(p, k, \tau)$ of a
polarity (either $+1$ or $-1$), the feature column index, and
threshold, respectively, which predicts a binary label from the set
$\{+1, -1\}$ for any input $x \in \mathbb{R}^K$ using the function
$h(x) \equiv p\mathop{\mathrm{sign}}(x[k] - \tau)$.

A decision tree $h_{\textit{Tree}}(x)$ is trained, top to bottom, by
``splitting'' a node, i.e. selecting a stump $h(x)$ that optimizes
some function such as error rate, information gain, or GINI impurity.  
While this chapter focuses on selecting stumps based on
error rate, we intend to extend our work to other measures.
Our algorithm \texttt{Adaptive-Pruning Stump} (Algorithm~\ref{adaptive-pruning-stump}), a subroutine of \texttt{Adaptive-Pruning Boost} (Algorithm~\ref{boosting}), trains a decision stump $h(x)$ with
fewer total example assessments than its analog, the subroutine of the-state-of-the-art algorithm
\texttt{Quick Boost}, does. Note that \texttt{Adaptive-Pruning Stump} used iteratively can train a decision tree, but for simplicity we assume our weak learners are binary decision stumps.
While we describe \texttt{Adaptive-Pruning Stump} for
binary classification, the reasoning also applies to multi-class data.
% the
%decision stump ultimately differentiates between one class and a group of
%classes (a binary classification).

To describe how \texttt{Adaptive-Pruning Stump} trains a stump
we need a few definitions.
Let $n$ be the total number of examples, and $m \le n$ some number of examples
on which a stump has been trained so far.
We will assume that Boosting provides the examples in decreasing weight
order.
This order can be maintained in $O(n)$ time in the presence of Boosting weight updates
because examples which are correctly classified do not change their relative
weight order, and examples which are incorrectly classified do not change their
relative weight order; a simple merge of these two groups suffices.
We can therefore number our examples from 1 to $n$ in decreasing weight order.
Furthermore,
\begin{itemize}[topsep=0pt,itemsep=0pt]
\item let $Z_m := \sum_{i=1}^m {w_i}$ be sum of the weights of first $m$ (heaviest) examples, and
\item let $\epsilon_m := \sum_{i=1}^m {w_i \mathbbm{1} \{h(x_i) \ne y_i\}}$ be the sum of the weights of the examples from the first $m$ which are misclassified by the stump $h(x)$.
\end{itemize}

The weighted error rate for stump $j$ on the first $m$ examples is then
$E^j_m := \epsilon_m^j / Z_m$.



